\documentclass[./\jobname.tex]{subfiles}
\begin{document}
%
	\blindmathtrue
	\blinddocument
	%
\chapter{Mathemodus}
%
\begin{align}
det(\lambda \cdot \mathds{1} -A) &= 0 \underbrace{\longrightarrow}_{\substack{\text{Zustands-} \\ \text{rückführung}}} det\left[\lambda\cdot \mathds{1} - \left(A-b\cdot k\right)\right]=0\label{eq: eigenwerte rueckfuehrung}\\
\sin(x)^{2} + \cos(x)^{2} &= 1\label{eq: cos sin}
\end{align}
%
Ich bin eine super Referenzierung: \cref{eq: eigenwerte rueckfuehrung}. Oder noch besser wenn auf mehrere Gleichungen referenziert werden will: \cref{eq: eigenwerte rueckfuehrung,eq: cos sin}
%
\chapter{BiB \& Acronyme}
%
\begin{itemize}
	\item Dies ist eine Qullenangabe: \parencite[Vgl.][S.220-224]{IEEE802.1Q2014}.
	\item Dies ist eine \gls{cbs}.
	\item Zweite Verwendung von einem Acronym: \gls{cbs}.
	\item Dies ist eine Fußzeile\footnote{Das ISO/OSI Model kann in \cite[][S.2-9]{Mandl2010} nachgeschlagen werden.}.
	\item \textcite[][S.2-9]{Mandl2010} nachgeschlagen werden.
\end{itemize}
%
\chapter{Bilder}
%
\begin{figure}[H]
\centering
\noindent\adjustbox{max width=\textwidth}{%falls größer als \textwidth, wird das Bild verkleinert
	%trim option's parameter order: left bottom right top
	\includegraphics[width=1\textwidth]{example-image-a}
}
	\unterschrift{ich bin die Unterschrift}{Quelle}{}
	\label{fig: example-image}
\end{figure}
%
\begin{figure}[H]
	\centering
	\begin{subfigure}[b]{0.5\textwidth}
		\centering
		\includegraphics[width=1\textwidth]{example-image-a}
		\caption{Bild A}
		\label{fig: Bild A}
	\end{subfigure}% 
	%
	\begin{subfigure}[b]{0.5\textwidth}
		\centering
		\includegraphics[width=1\textwidth]{example-image-b}
		\caption{Bild B}
		\label{fig: Bild B}
	\end{subfigure}%
	\unterschrift{Bildunterschrift für beide Bilder}{\cite{Dorner2010}}{}%
	\label{fig: Bild A und B}
\end{figure}
%
\section{Spezial}
%
\def\spaceX{1.5}
\begin{figure}[H]
	\centering
\begin{tikzpicture}%[scale=0.7]
\begin{scope}[every node/.style={bgelement}]
\node (Se) at (0,0) {Se};
\node[right=\spaceX of Se] (i) {1};
\node[above=\spaceX of i] (Iel) {$I:L_{a}$};
\node[below=\spaceX of i] (Rel) {$R:R_{a}$};
\node[right=\spaceX of i,label=below:{$\varPsi=k_{m}$}] (GY) {GY};
\node[right=\spaceX of GY] (w) {1};
\node[above=\spaceX of w] (Im) {$I:J_{m}$};
\node[below=\spaceX of w] (Rm) {$R:b_{m}$};
\end{scope}
\draw[bonds]
(Se) edge [e_out,effort={$U$}, flow={$i_a$}] (i)
(i) edge [e_out,effort={$U_{La}$}, flow={$i_a$}] (Iel)
edge [e_in,effort={$U_{Ra}$}, flow={$i_a$}] (Rel)
edge [e_in,effort={$U_{emf}$}, flow={$i_a$}] (GY)
(GY) edge [e_out,effort={$T$}, flow={$\omega$}] (w)
(w) edge [e_out,effort={$T_{j}$}, flow={$\omega$}] (Im)
edge [e_in,effort={$T_{b}$}, flow={$\omega$}] (Rm);
\end{tikzpicture}
	\unterschrift{Kausalisierter Bondgraph}{eigene Ausarbeitung}{}
\label{fig: kausalisierter bondgraph}
\end{figure}
%
\def\bildA{true}
\begin{figure}[H]
	\centering
	\noindent\adjustbox{max width=\textwidth}{%falls größer als \textwidth, wird das Bild verkleinert
	\subfile{./img/tikz/escon-schaltplan.tex}
}
	\unterschrift{Schaltkreis: Escon 50/5 Controller für Polverschiebung}{eigene Ausarbeitung}{}
	\label{fig: Schalplan Polverschiebung}
\end{figure}
%
\def\bildA{true}
\begin{figure}[H]
	\centering
	\noindent\adjustbox{max width=\textwidth}{%falls größer als \textwidth, wird das Bild verkleinert
		\subfile{./img/tikz/blockschaltbild-state-space.tex}
	}
	\unterschrift{Zustandsregler mit Rückkopplungsverstärkungsmatrix f}{eigene Ausarbeitung}{}
	\label{fig: blockschaltbild-state-space.tex A}
\end{figure}
%
\chapter{Tabellen}
%
\begin{table}[H]
	\centering
	\noindent\adjustbox{max width=\linewidth}{
		\begin{tabular}{|c|c|c|c|c|c|c|c|c|c|}
			\hline
			\rowcolor[HTML]{\farbeTabA}
			Switch Typ & load & $n$ & min & max & $\tilde{x}$ & $\bar{x}$ & $\sigma^{2}$ & $\sigma$ \\ \hline
			Akro 6/0 & nein & 1199602 & 145.12 & 151.32 & 147.92 & 147.91 & 1.12 & 1.06	\\ \hline
			\rowcolor[HTML]{\farbeTabB} 
			Akro 6/0 & ja & 1199382 & 145.12 & 151.40 & 147.88 & 147.89 & 1.13 & 1.06 \\ \hline
		\end{tabular}
	}
	\unterschrift{Tabellen Unterschrift}{eigene Ausarbeitung}{}
	\label{tab: Tabelle A}
\end{table}
%
\chapter{Listing}
%
\begin{lstlisting}[caption={example}]
for i:=maxint to 0 do
begin
	{ comment } %(*\label{comment}*)
end;
\end{lstlisting}
Line \ref{comment} shows a comment. Die Escape Operatoren können unter \enquote{./sty/Listings.sty} geändert werden.
%
\end{document}
